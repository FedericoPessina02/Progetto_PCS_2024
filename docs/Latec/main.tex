\documentclass[a4paper,12pt]{report}
%%%%%%%%%%%%%%%%%%%%%%%%%%%%%%%%%%%%%%%%%%%%%%%%%%%%

\usepackage[utf8]{inputenc}
\usepackage[T1]{fontenc}
\usepackage{graphicx}
\usepackage{microtype}
\usepackage{amsmath}
\usepackage{amsthm}
\usepackage{amssymb}
\usepackage{lmodern}
\usepackage{hyperref}
\usepackage{geometry}
\hypersetup{%
    pdfpagemode={UseOutlines},
    bookmarksopen,
    pdfstartview={FitH},
    colorlinks,
    linkcolor={blue},
    citecolor={blue},
    urlcolor={blue}
  }
\newtheorem{theorem}{Teorema}
\newtheorem{lemma}{Lemma}
\newtheorem{corollary}{Corollario}
\newtheorem{definition}{Definizione}
  
\begin{document}\errorcontextlines=9 %numero massimo di linee che hanno un errore

  
\begin{titlepage}
\newgeometry{left=1cm,right=1cm,top=3cm,bottom=3.5cm}  %specific margins for this page

\begin{center}

{\huge POLITECNICO DI TORINO}\\[1.5cm]
\textbf{Corso di Laurea\\in Matematica per l'Ingegneria}\\[3cm]
%\textbf{Corso di Laurea Magistrale\\in Ingegneria Matematica}\\[3cm]

{\Large Progetto di}\\[1cm]
%{\Large Tesi di Laurea Magistrale}\\[0.5cm]
\textbf{\LARGE Programmazione e calcolo scientifico }\\[2cm]
\includegraphics[width=0.26\textwidth]{polito_logo}
\vspace{4cm}

%\customfont{hello world!}

\begin{minipage}{0.85\textwidth}
\begin{flushleft}\large
\textbf{Partecipanti} \hfill \textbf{Matricole}\\
Emanuele Martin \hfill 296214 \\
Andrea Terenziani \hfill 284817 \\
Federico Pessina \hfill 293945 \\
% \fillin\ \hfill \\
% \fillin\ \hfill \fillin
\end{flushleft}
\end{minipage}

\vfill

Anno Accademico 2023-2024
\end{center}

\restoregeometry %restore default margins 

\end{titlepage}

\chapter*{Introduzione}
Il progetto assegnato fornisce in input un DFN (Discrete Fracture Network, ovvero un sistema di poligoni planari, detti fratture, intersecati tra di loro) e richiede due consegne differenti:

innanzitutto di individuare e catalogare le intersezioni tra i poligoni, dette tracce, in passanti e interne, e secondariamente di tagliare i poligoni seguendo le tracce stesse e utilizzando come gerarchia di taglio prima le tracce passanti (ordinate per lunghezza decrescente) e successivamente quelle interne (sempre in ordine di lunghezza decrescente).


Per progettare e sviluppare il codice risolutivo, ci siamo soffermati su 3 fasi principali, che hanno reso possibile in primis il raggiungimento degli output richiesti dalla consegna, e in secondo luogo hanno reso il programma più facilmente leggibile e computazionalmente efficace (e di conseguenza dunque, ottimale in ottica di tempistiche risolutive):

-organizzazione del codice: ovvero la suddivisione su più file e varie classi, per rendere il lavoro più leggibile, ordinato e lineare da programmare.

-scelte logiche e procedimento: ovvero la fase di decisione su come approcciarsi al problema stesso e quali meccanismi risolutivi fossero necessari in ciascuna funzione utilizzata.

-ottimizzazione algoritmo e strutture dati: ovvero una fase più tecnica, in cui vengono presi numerosi accorgimenti per rendere il codice performante, rapido e computazionalmente vantaggioso.


\tableofcontents
\chapter{Struttura del codice}


Il programma utilizza 3 classi principali: \newline \newline
\textbf{-Fracture:}
E' la classe di una singola frattura, contenente tutte le sue caratteristiche e i rispettivi metodi.
Oltre alle ovvie informazioni quali id, matrice dei vertici ecc... in essa sono presenti: \*internal\_traces e passant\_traces : contenente gli id delle tracce di tale frattura (rispettivamenete interne nel primo caso e passanti nel secondo);

partition\_id : ovvero la partizione dello spazio a cui appartiene la frattura.

Tale informazione risulta fondamentale per ridurre i tempi di calcolo (e verrà spiegata più nel dettaglio nella terza sezione della relazione)

\textbf{-TracesMesh:}

E' l'elenco di tutte le tracce

Le informazioni contenute sono ad esempio gli Id della traccia, la lunghezza, le fratture che l'hanno generata, le coordinate dei vertici estremi, ecc

\textbf{-PolygonalMesh:}

E' la mesh di ogni singola frattura.

Risulta fondamentale per per il secondo punto del progetto, in quanto in essa vengono salvati tutti i nuovi poligoni generati dai tagli.

Come di consuetudine, è stata divisa in celle 0Ds ovvero i vertici, 1Ds ovvero i lati e 2Ds ovvero i poligoni.


Il codice (e di conseguenza quindi anche le funzioni) è invece stato diviso su più file:

\textbf{-main.cpp:}

Il suo scopo principale è quello di chiamare le funzioni (attivando così una cascata di chiamate ricorsive) e di calcolare i tempi impiegati dal codice (fondamentale per confermare che le scelte di ottimizzazione prese, fossero effettivamente vantaggiose)


\textbf{-Algorithms.cpp:}

In esso sono contenute tutte le funzioni che operano sulle fratture e sulle tracce, in grado di fornire i risultati concreti ricercati e necessari per l'avanzamento dell'algoritmo, a partire dalle strutture dati già munite degli input necessari


\textbf{-Utils.cpp:}

contiene le funzioni necessarie per il funzionamento del programma, più slegate tutta via dal procedimento logico puramente inerente allo studio delle fratture e delle tracce. Ne sono un esempio le funzioni di stampa, di lettura e salvataggio dei dati, di comparazione...

\textbf{-Test.hpp:}

in questo file viene testata la correttezza di tutte le principali funzioni utilizzate. E' di fondamentale importanza in quanto permette di concentrarsi esclusivamente sulla progettazione logica del problema, escludendo i casi in cui il procedimento è giusto ma l'output ottenuto è sbagliato.



\subsection{Scelte logiche e procedimento}


\textbf{-consegna numero 1:}


Per quanto riguarda la risoluzione della prima richiesta del progetto, la porzione di codice che ha fornito il maggiore contributo, richiedendo così numerose scelte sia a livello procedurale che implementativo, è collocata all'interno di Fracture.cpp e più precisamente è il metodo \textit{GenerateTrace}. \newline


Tale metodo infatti viene applicato su una frattura, e consiste nel confrontarla con un'altra frattura appartenente alla sua stessa partizione e con il baricentro sufficientemente vicino, calcolare la traccia generata da esse e infine distinguerne la natura (passante o interna).

Per fare ciò sono necessari più passaggi e la divisione su più casistiche:

Cercare l'equazioni dei piani su cui giacciono le fratture, e calcolarne l'intersezione che sarà dunque una retta (a meno di fratture parallele)

Cercare l'intersezione tra tale retta e le due fratture, e trovare quanti di questi punti siano DISTINTI. In base al numero di punti ottenuti, siamo in grado di catalogare il tipo di traccia:

2: i punti di intersezione coincidono a coppie e la traccia è passante per entrambe le fratture

3: un punto di intersezione è in comune per entrambe le fratture. Trovato quel punto, il segmento di lunghezza minore sarà la traccia, e risulterà passante per la frattura che ha entrambi i vertici della traccia sul bordo, e interna per l'altra

4: Innanzitutto cerco il punto estremo, e i due punti più vicino ad esso, e poi catalogo per tutti i casi che possono capitare:

il segmento di intersezione è esterno ad entrambi i poligoni: non c'è traccia

se i due punti vicini (e non l'estremo) appartengono alla stessa frattura, la traccia sarà il segmento di unione tra questi due punti, e sarà passante per quest'ultima e interna per l'altra.

se i due punti vicini (e non l'estremo) NON appartengono alla stessa frattura, la traccia sarà il segmento di unione tra questi due punti, e sarà interna per entrambe le fratture \newline

\textbf{-consegna numero 2:}


Per quanto riguarda la risoluzione della seconda richiesta del progetto invece, il metodo che contiene il procedimento risolutivo, è sempre contenuto in Fracture.cpp e si chiama

CutPolygonalMesh.


Tale metodo permette di, data una frattura, tagliarla progressivamente in più fratture seguendo le tracce (rigorosamente effettuando i tagli prima lungo le tracce passanti e poi quelle interne, ordinate entrambe per lunghezza decrescente) e aggiornare la mesh con i dati delle fratture ottenute.

Per fare ciò, a partire da una mesh contenente delle fratture (attive e non), si riconduce ricorsivamente a casistiche sempre più facili da studiare attraverso alle chiamate del metodo CutMeshBySegment e della funzione Algorithms::CutPolygonBySegment.


Analizzandola più nel dettaglio, le casistiche da gestire e gli accorgimenti da adottare per potersi ricondurre al caso più banale di un singolo poligono tagliato da una traccia passante, sono raggruppabili in tale modo:

mesh costituita da una sola frattura attiva (primo step) o da più fratture, alcune attive e altre disattivate (step successivi)

operazione di taglio lanciata su una traccia passante (caso banale) o traccia interna (caso più complicato).

Nel caso di una traccia interna, sarà necessario ricondursi ad una traccia passante:

viene prolungata la traccia

si cerca l'intersezione tra la retta su cui giace la traccia e ogni lato del poligono

si verifica quali coefficienti trovati dalla soluzione di tale sistema siano compresi tra 0 e 1 (ciò garantisce che la combinazione sia convessa e l'intersezione avvenga all'interno del lato e non sul suo prolungamento)

si verifica che venga tagliato solo uno dei poligoni attivi

Infine, arrivati al caso banale, CutPolygonBySegments effettua il taglio e la creazione di due nuove fratture attive, disattivando la frattura padre precedente, e garantisce un'uniformità nell'ordinamento dei vertici (vecchi e nuovi) di ciascuna frattura generata.
\chapter{Scelte implementative}

\section{Scelte computazionali per le principali funzioni}
Per quanto riguarda l'efficienza in termini di spazio l'utilizzo delle referenze per passare gli argomenti delle funzioni è fondamentale per evitare di generare, inutilmente, delle copie ad ogni chiamata di funzione. \newline
Invece i \textit{const} prima di passare gli argomenti sono stati utilizzati quando il dato passato non doveva venire modificato in quella funzione. \newline
E' stato necessario ordinare le fratture in ordine decrescente per lunghezza, e per tale scopo abbiamo incluso gli algoritmi di ordinamento \textit{Mergesort} e \textit{Bubblesort}. \newline
Nonostante il costo computazionale del Bubblesort sia $\mathcal{O}(n^2)$ che risulta essere maggiore del costo per implementare il Mergesort che invece è $\mathcal{O}(n \log n) $, per i dataset forniti impiega meno tempo. Questo a causa del basso costo fisso del Bubblesort, e di come funzionano i due algoritmi: il Mergesort (a differenza del Bubblesort) è basato sulle chiamate di funzioni ricorsive, ciascuna delle quali ha un impatto sulla velocità dell' algoritmo.


\section{Le classi utilizzate}
\textit{Fracture} - La prima classe che abbiamo introdotto è quella per memorizzare le fratture; abbiamo optato per salvare una frattura come istanza della classe, piuttosto che avere come istanza la lista di tutte le fratture, per semplificare l'accesso e la definizione dei metodi sulle istanze. \newline
\textit{TracesMesh} - Per invece esportare le informazioni riguardanti le tracce, abbiamo utilizzato l'approccio opposto memorizzando in una singola istanza della classe tutte le tracce calcolate, dal momento in cui non erano richiesti particolari conti. \newline
\textit{PolygonalMesh} - Nella seconda parte del progetto abbiamo invece costruito una mesh per ogni frattura, che contiene tutti i poligoni ottenuti tramite i tagli delle tracce. Questo approccio permette di suddividere il problema globale, dei tagli delle fratture, in diversi problemi locali indipendenti tra loro.


\section{Strutture dati}
I due principali tipi di contenitori impegiati sono stati i vettori della libreria standard, utili quando le loro dimensioni non erano note a priori, ed i vettori/matrici della libreria \textit{Eigen}, che hanno il vantaggio di avere implementati nativamente i metodi per svolgere le operazioni tra vettori e matrici, come ad esempio prodotti vettoriali e risoluzioni di problemi lineari. \newline
Abbiamo utilizzato in alcune funzioni la struttura map (ad esempio per associare ad una frattura tutte le tracce coinvolte).

\section{Scelta della tolleranza e altri dettagli}
Poichè è stato necessario effettuare diversi calcoli durante il progetto, spesso ci si è posto davanti il problema di dover stabilire quando due numeri fossero uguali, e poichè i conti sul calcolatore non sono esatti, abbiamo dovuto tenere in considerazione che le uguaglianze in senso stretto non hanno senso, e dunque abbiamo introdotto le uguaglianze a meno di una tolleranza.
Come valore per la tolleranza abbiamo scelto il valore $500$*$\epsilon$ con $ \epsilon =$ epsilon di macchina. \newline
Il valore è stato scelto euristicamente grande a sufficienza per evitare di classificare due punti diversi come lo stesso, e piccolo abbastanza per sfruttare il più possibile la precisione fornita dai dati iniziali.  Inoltre è stato inserito come variabile universale quindi facilmente modificabile\newline
\newline
\textbf{Altri dettagli:} \newline
\begin{itemize}
    \item Abbiamo inoltre dovuto stabilire se alcuni segmenti fossero degeneri (distanza tra gli estremi troppo piccola) e per farlo abbiamo considerato anzichè la norma dei vettori la norma al quadrato evitando così di chiamare la funzione \textit{norm} che include la radice quadrata, che è una funzione computazionalmente costosa.
    %\item Nella risoluzione dei sistemi lineari, utilizzata nei calcoli relativi ad intersezioni tra piani/rette, i sistemi sono stati risolti 5con le fattorizzazioni PA-LU o QR rispettivamente per matrici dei coefficienti quadrate, con costo computazionale $\mathcal{O}(\frac{1}{3}n^3)$ , e rettangolari, con costo $\mathcal{O}(\frac{2}{3}n^3) $
    \item In alcune funzioni avremmo potuto utilizzare il comando inline per velocizzare il codice: \textit{addPoint} e \textit{addEdge} sono due metodi per i quali, il comando inline, sarebbe stato opportuno in quanto entrambi sono brevi e vengono richiamati di sovente. \newline Inline evita che tali metodi vengano richiamati ogni volta, alleggerendo così lo stack frame.
    \item Nella seconda parte del progetto inizialmente avevamo individuato un altro algoritmo per risolvere il problema dei tagli delle fratture, ma procedendo nel suo sviluppo ci siamo resi conto che era molto dispendioso, in quanto si basava sul calcolo di molti prodotti vettoriali e di molti sistemi lineari, e lo abbiamo dunque accantonato per prendere in considerazione un'altra strada (che è risultata essere piuttosto efficiente).
 
\end{itemize}



\end{document}